% OM-2
\noindent
\begin{tabular}%
       {|>{\colleft}p{3cm}%
        |>{\colleft}p{10cm}|}
\hline
{\bf Organization Model} &
   {\bf Variant Aspects Worksheet OM-2} \\
\hline
\hline
{\sc Structure} &
% Give an organization chart of the considered (part of the) organization
% in terms of its departments, groups, units, sections, ...
The organizational structure is simple. The car hobbyist is working alone.
\\
\hline
{\sc Process} &
% Sketch the layout (e.g., with the help of a UML activity diagram)
% of the business
% process at hand. A process is the relevant part of the value
% chain that is focused upon. A process is
% decomposed into tasks, which are detailed in worksheet OM-3.
There is the process of repairing a car.
\\
\hline
{\sc People} &
% Indicate which staff members are involved, as actors or
% stakeholders, including decision makers, providers, users or
% beneficiaries (``customers'') of knowledge. These people do not need
% to be actual people, but can be functional roles played by people in
% the organization (e.g., director, consultant)
The car hobbyist might occasionally have an assistant if he perform a task that
requires it. Like checking whether the headlights are working.
\\
\hline
{\sc Resources} &
% Describe the resources that are utilized for the business
% process. These may cover different types, such as:
 % 1. Information systems and other computing resources
 Repair manuals, technical description of a specific car, general technical
description of car.
 \\
 % 2. Equipment and materials
 &
 Tools.
 \\
 % 3. Technology, patents, rights
% &
% \\
\hline
{\sc Knowledge} &
% Knowledge represents a special resource exploited in a business
% process. Because of its key importance in the present context, it
% is set apart here. The description of this component of the
% organization model is given separately, in worksheet OM-4 on
% knowledge assets.
The car hobbyist uses knowledge about how to repair a car, knowledge about how
cars work in general and knowledge about how a specific car works. \\
\hline
{Culture \& power} &
% Pay attention to the unwritten rules of the game,
% including styles of working and communicating (``the way we do
% things around here''), related social and interpersonal
% (non-knowledge) skills, and formal as well as informal
% relationships and networks.
Car repair happens in the social environment of his or her family.
\\
\hline
\end{tabular}
