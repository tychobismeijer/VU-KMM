\noindent
\begin{tabular}%
       {|>{\colleft}p{3cm}%
        |>{\colleft}p{10cm}|}
\hline
{\bf Organization Model} &
  {\bf Checklist for Feasibility Decision Document: Worksheet OM-5} \\
\hline
\hline
{\sc Business feasibility} &
% For a given problem/opportunity area and a suggested solution, the
% following questions have to be answered:

% 1. What are the expected benefits for the organization
%           from the considered solution? Both tangible economic and
%           intangible business benefits should be identified here.
We expect that: car repairs are performed more quickly, car repairs with a higher difficulty can
be performed and the hobbyist learns more while repairing his car. The car
hobbyist spends less time on thinking about repairs and performing repairs that
don't result in a repaired car. Depending on the hobbyist this might increase or
decrease his fun in repairing the car.
\\
% 2. How large is this expected added value?
&
We don't know how much time might be saved, or how much more a hobbyist might
learn. 
\\
% 3. What are the expected costs for the considered solution?
& The hobbyist does need to have computer to run the knowledge system on. That
computer also needs an interface that is usable with greasy hands.\\
% 4. How does this compare to possible alternative solutions?
&
This compare favourably with the other solutions. The hobbyist would be prepared
to spend money on a knowledge system, while the time needed for further eduction
is not available, and a knowledge retrieval system would not have the same
benefits. \\
% 5. Are organizational changes required?
& There is no need for organizational changes. The hobbyist still works on his
own while repairing the car. \\
% 6. To what extent are economic and business risks and
%           uncertainties involved regarding the considered solution
%           direction?
%&
%\\
\hline
\sc Technical feasibility &
%   For a given problem/opportunity area and a suggested solution, the
%   following questions have to be answered:
% 1. How complex, in terms of knowledge stored and
%           reasoning processes to be carried out, is the task to be
%           performed by the considered knowledge-system solution?
%           Are state-of-the-art methods and techniques available and
%           adequate?
The system needs to perform diagnosis with causal reasoning. That can be done
with a knowledge system. There is no need to reason with the knowledge about how
to repair a car, this can be represented in text form to the user.
\\
& 
% 2. Are there critical aspects involved, relating to time,
%           quality, needed resources, or otherwise? If so, how to
%           go about them?
The system needs to run on a PC sized computer. It should respond in seconds to
response of the user. The system has minutes of reasoning time when the hobbyist
is performing repairs.
\\
&
%Is it clear what the success measures are and how to test
%           for validity, quality, and satisfactory performance?
The system is successful if it can guide a car repair hobbyist in common repairs
to the electrical system of one specific car. It also needs to be modular and
modifiable enough, so that it's clear it can be extended to more special
repairs, repairing more kinds of car and repairing non-electrical faults.
\\
%&
%4. How complex is the required interaction with end users
%           (user interfaces)? Are state-of-the-art methods and
%           techniques available and adequate?
%\\
%& 
% How complex is the interaction with other information
%           systems and possible other resources (interoperablity,
%           systems integration)? Are state-of-the-art methods and
%           techniques available and adequate?
%\\
%&
% 6. Are there further technological risks and uncertainties?
%\\
\hline
Project feasibility &
%   For a given problem/opportunity area and a suggested solution, the
%   following questions have to be answered:
The project is feasible. We have access to an expert on car repair and a
car hobbyist. 
\\
\hline
\end{tabular}
